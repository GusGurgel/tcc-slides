%%%%%%%%%%%%%%%%%%%%%%%%%%%%%%%%%%%%%%%%%%%%%%%%%%%%%%%%%%%%%%%%%%%%%%%%%%%%%%%%%%%
%% This project aims to create the UFC template for presentation.                %%
%% author: Maurício Moreira Neto - Doctoral student in Computer Science (MDCC)   %%
%% contacts:                                                                     %%
%%    e-mail: maumneto@ufc.br                                                    %%
%%    linktree: https://linktr.ee/maumneto                                       %%
%%%%%%%%%%%%%%%%%%%%%%%%%%%%%%%%%%%%%%%%%%%%%%%%%%%%%%%%%%%%%%%%%%%%%%%%%%%%%%%%%%%
\documentclass{libs/ufc_format}
% Inserting the preamble file with the packages
\input{libs/preamble.tex}
% Inserting the references file
\bibliography{references.bib}

% Title
% \title[Geração de Mapas com LLMs]{\huge\textbf{Geração de Mapas com LLMs para Roguelikes}}
% Subtitle
\subtitle{Geração de Mapas para Jogos Roguelike a Partir de Descrições Textuais
Utilizando Modelos de Linguagem de Larga Escala }
% Author of the presentation
\author{Gustavo Gurgel (Orientador: Cristiano Bacelar)}
% Institute's Name
\institute[UFC]{
    % email for contact
    \normalsize{\email{gusgurgel@alu.ufc.br}}
    \newline
    % Department Name
    % \department{}
    % \newline
    % university name
    \ufc
}
% date of the presentation
\date{\today}


%%%%%%%%%%%%%%%%%%%%%%%%%%%%%%%%%%%%%%%%%%%%%%%%%%%%%%%%%%%%%%%%%%%%%%%%%%%%%%%%%%
%% Start Document of the Presentation                                           %%               
%%%%%%%%%%%%%%%%%%%%%%%%%%%%%%%%%%%%%%%%%%%%%%%%%%%%%%%%%%%%%%%%%%%%%%%%%%%%%%%%%%
\begin{document}
% insert the code style
\input{libs/code_style}

%% ---------------------------------------------------------------------------
% First frame (with tile, subtitle, ...)
\begin{frame}{}
    \maketitle
\end{frame}

%% ---------------------------------------------------------------------------
% Second frame
\begin{frame}{Sumário}
    \begin{multicols}{2}
        \tableofcontents
    \end{multicols}
\end{frame}

%% ---------------------------------------------------------------------------
% This presentation is separated by sections and subsections
% \section{Seção I}
% \begin{frame}{Explicações}
%     % itemize
%     Este é um template que pode ser utilizado para \nocite{}:
%     \begin{itemize}
%         \item Apresentação de Trabalhos Acadêmicos
%         \item Apresentação de Disciplinas
%         \item Apresentações de Teses e Dissertações
%     \end{itemize}

%     \vspace{0.4cm} % vertical space

%     % enumeration
%     Para utilizar este template corretamente é importante que:
%     \begin{enumerate}
%         \item Tenha conhecimento mínimo sobre LaTeX
%         \item Ler os comentários no template (explicações)
%         \item Ler o README.md (documentação)
%     \end{enumerate}

%     \vspace{0.2cm}

%     \example{Este é um texto de exemplo!} \emph{Texto de Ênfase!}
% \end{frame}

% %% ---------------------------------------------------------------------------
% \subsection{Subseção I}
% \begin{frame}{Criando Blocos}
%     % Blocks styles
%     \begin{block}{Bloco Padrão}
%         Texto do corpo do bloco.
%     \end{block}

%     \begin{alertblock}{Bloco de Alerta}
%         Texto do corpo do bloco.
%     \end{alertblock}

%     \begin{exampleblock}{Bloco de Exemplo}
%         Texto do corpo do bloco.
%     \end{exampleblock}
% \end{frame}

% %% ---------------------------------------------------------------------------
% \subsection{Subseção II}
% \begin{frame}{Criando Caixas}
%     \successbox{testando o success box}

%     \pause

%     \alertbox{testando o alert box}

%     \pause

%     \simplebox{testando o simple box}
% \end{frame}

% %% ---------------------------------------------------------------------------
% \subsection{Subseção III}
% \begin{frame}{Criando Algoritmos (Pseudocódigo)}
%     \begin{algorithm}[H]
%         \SetAlgoLined
%         \LinesNumbered
%         \SetKwInOut{Input}{input}
%         \SetKwInOut{Output}{output}
%         \Input{x: float, y: float}
%         \Output{r: float}
%         \While{True}{
%             r = x + y\;
%             \eIf{r >= 30}{
%                 ``O valor de $r$ é maior ou iqual a 10.''\;
%                 break\;
%             }{
%                 ``O valor de $r$ = '', r\;
%             }
%         }
%         \caption{Algorithm Example}
%     \end{algorithm}
% \end{frame}

% %% ---------------------------------------------------------------------------

% \begin{frame}{Inserindo Algoritmos}
%     \lstset{language=Python}
%     \lstinputlisting[language=Python]{code/main.py}
% \end{frame}

% %% ---------------------------------------------------------------------------
% \begin{frame}{Inserindo Algoritmos}
%     \lstinputlisting[language=C]{code/source.c}
% \end{frame}

% %% ---------------------------------------------------------------------------
% \begin{frame}{Inserindo Algoritmos}
%     \lstinputlisting[language=Java]{code/helloworld.java}
% \end{frame}

% %% ---------------------------------------------------------------------------
% \begin{frame}{Inserindo Algoritmos}
%     \lstinputlisting[language=HTML]{code/index.html}
% \end{frame}

%% ---------------------------------------------------------------------------
% This frame show an example to insert multicolumns
\section{Introdução}
\begin{frame}{Panorama Atual do Mercado de Jogos}

    \begin{columns}{}
        \begin{column}{0.6\textwidth}
            \begin{figure}
                \centering
                \caption{Global games market: growth drivers and challenges for 2025-2027}
                \includegraphics[width=6cm]{images/games_2015_2027_revenue.png}
                \source{Newzoo}
            \end{figure}
        \end{column}
        \begin{column}{0.4\textwidth}
            \justify
            \small{O gráfico demonstra o crescimento constante do mercado global de
                jogos de 2015 a 2027, com uma previsão de aumento de 3.7\% da sua
                receita entre 2025 e 2027. Com uma renda prevista de \textbf{\$198
                    bilhões} em 2027, evidencia a importância desse mercado na economia
                atual.}
        \end{column}
    \end{columns}
\end{frame}

\begin{frame}{Crescimentos dos Jogos Roguelikes}
    \begin{columns}{}
        \begin{column}{0.6\textwidth}
            \begin{figure}
                \centering
                \caption{Balatro}
                \includegraphics[width=6cm]{images/balatro.jpg}
                \source{Wikipedia}
            \end{figure}
        \end{column}
        \begin{column}{0.4\textwidth}
            \justify
            \small{Balatro é um roguelike de construção de baralhos. \textbf{Foi um dos
                    6 jogos indicados para jogo do ano no The Game Awards 2024}. Isso
                mostra que, mesmo sendo um gênero antigo, ainda possui grande
                destaque no mercado de jogos.}
        \end{column}
    \end{columns}
\end{frame}

\begin{frame}{Crescimentos dos Jogos Roguelikes}
    \begin{figure}
        \centering
        \caption{Hell Clock (2025): Roguelike brasileiro inspirado na Guerra dos Canudos}
        \includegraphics[width=10cm]{images/hellclock.png}
        \source{Steam}
        \label{fig:ufc_emblem}
    \end{figure}
\end{frame}


\begin{frame}{Geração Procedural de Conteúdo (PCG)}
    \begin{figure}
        \centering
        \caption{Mapas gerados utilizando PCG (The Binding of Isaac)}
        \includegraphics[width=9cm]{images/bind_of_issac_maps.png}
        \source{Level Generation by Joining Geometry \cite{e2020procedural}}
    \end{figure}
    \small{Uma das principais características dos jogos Roguelike é a presença
        de conteúdos gerados proceduralmente. Ou seja, conteúdos gerados por
        algoritmos, geralmente utilizam geradores de números pseudo aleatórios.}
\end{frame}

\begin{frame}{Geração Procedural de Conteúdo (PCG)}
    \begin{columns}{}
        \begin{column}{0.6\textwidth}
            \begin{figure}
                \centering
                \caption{Gerador de Mapas com Diversos Parâmetros de Configuração}
                \includegraphics[width=6cm]{images/pcg_config.png}
                \source{Regressão para Predição de Mapas Gerados Proceduralmente. \cite{marchandaplicaccao}}
            \end{figure}
        \end{column}
        \begin{column}{0.4\textwidth}
            \justify
            \small{A geração desses conteúdos geralmente pode ser controlada por
                parâmetros como tamanho do objeto gerado, quantidades de elementos ou até
                sementes para o gerador pseudo aleatório.}
        \end{column}
    \end{columns}
\end{frame}

\begin{frame}{PCG + LLMs}
    \begin{figure}
        \centering
        \caption{Arquitetura do MarioGPT}
        \includegraphics[width=6cm]{images/mario_gpt.png}
        \source{MarioGPT: Open-Ended Text2Level Generation through Large Language Models \cite{sudhakaran2023mariogpt}}
    \end{figure}
    \small{Com o surgimento dos Modelos Transformers Generativos, surgem
        grandes oportunidades de integrar essa tecnologia na geração
        procedural de conteúdo em jogos. Um exemplo disso é o \textbf{MarioGPT},
        projeto que \textbf{utiliza um modelo ajustado do GPT 2 para gerar mapas do
            jogo Mario}. Nesse projeto, o controle da geração é feito por prompts
        textuais, tornando a geração mais expressiva do que a configuração
        de parâmetros predefinidos.  }
\end{frame}

\begin{frame}{Objetivo Geral}
    \begin{itemize}
        \item<2-> Desenvolver um sistema que recebe descrições textuais e, utilizando um LLM, traduz em um nível estruturado e jogável de um jogo no estilo roguelike
    \end{itemize}
\end{frame}

\begin{frame}{Objetivos Específicos}
    \begin{itemize}
        \item<2-> Investigar e aplicar técnicas de engenharia de prompt para guiar o
              LLM de forma eficaz, assegurando que os mapas gerados sejam coerentes
              com a descrição textual e respeitem as restrições e convenções do gênero
              roguelike.
        \item<3-> Desenvolver um pipeline de geração que receba uma descrição
              textual como entrada e utilize um LLM para produzir uma representação de
              mapa estruturada, contendo elementos como salas, corredores, inimigos e
              itens.
        \item<4-> Implementar um protótipo funcional que integre o pipeline de
              geração, capaz de processar os dados do mapa e renderizá-los
              visualmente, servindo como prova de conceito da abordagem "text-to-map"
    \end{itemize}
\end{frame}

%% ---------------------------------------------------------------------------
% This frame show an example to insert figures
\section{Imagens}
\begin{frame}{Seção III - Figures}
    \begin{figure}
        \centering
        \caption{Emblema da UFC.}
        \includegraphics[scale=0.3]{libs/emblemufc.pdf}
        \source{Obtido pelo site oficial da UFC \cite{siteufc} \cite{einstein}}
        \label{fig:ufc_emblem}
    \end{figure}


\end{frame}

%% ---------------------------------------------------------------------------
% Reference frames
\begin{frame}[allowframebreaks]
    \frametitle{Referências}
    \printbibliography
\end{frame}

% %% ---------------------------------------------------------------------------
% % Final frame
% \begin{frame}{}
%     \centering
%     \huge{\textbf{\example{Obrigado(a) pela Atenção!}}}

%     \vspace{1cm}

%     \Large{\textbf{Contato:}}
%     \newline
%     \vspace*{0.5cm}
%     \large{\email{usuario@dominio}}
% \end{frame}

\end{document}
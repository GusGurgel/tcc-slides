%%%%%%%%%%%%%%%%%%%%%%%%%%%%%%%%%%%%%%%%%%%%%%%%%%%%%%%%%%%%%%%%%%%%%%%%%%%%%%%%%%%
%% This project aims to create the UFC template for presentation.                %%
%% author: Maurício Moreira Neto - Doctoral student in Computer Science (MDCC)   %%
%% contacts:                                                                     %%
%%    e-mail: maumneto@ufc.br                                                    %%
%%    linktree: https://linktr.ee/maumneto                                       %%
%%%%%%%%%%%%%%%%%%%%%%%%%%%%%%%%%%%%%%%%%%%%%%%%%%%%%%%%%%%%%%%%%%%%%%%%%%%%%%%%%%%
\documentclass{libs/ufc_format}
% Inserting the preamble file with the packages
\input{libs/preamble.tex}
% Inserting the references file
\bibliography{references.bib}

% Title
% \title[Geração de Mapas com LLMs]{\huge\textbf{Geração de Mapas com LLMs para Roguelikes}}
% Subtitle
\subtitle{Geração de Mapas para Jogos Roguelike a Partir de Descrições Textuais
Utilizando Modelos de Linguagem de Larga Escala }
% Author of the presentation
\author{Gustavo Gurgel (Orientador: Cristiano Bacelar)}
% Institute's Name
\institute[UFC]{
    % email for contact
    \normalsize{\email{gusgurgel@alu.ufc.br}}
    \newline
    % Department Name
    % \department{}
    % \newline
    % university name
    \ufc
}
% date of the presentation
\date{\today}


%%%%%%%%%%%%%%%%%%%%%%%%%%%%%%%%%%%%%%%%%%%%%%%%%%%%%%%%%%%%%%%%%%%%%%%%%%%%%%%%%%
%% Start Document of the Presentation                                           %%               
%%%%%%%%%%%%%%%%%%%%%%%%%%%%%%%%%%%%%%%%%%%%%%%%%%%%%%%%%%%%%%%%%%%%%%%%%%%%%%%%%%
\begin{document}
% insert the code style
\input{libs/code_style}

\begin{frame}{}
    \maketitle
\end{frame}

\begin{frame}{Sumário}
    \begin{multicols}{2}
        \tableofcontents
    \end{multicols}
\end{frame}

%% ---------------------------------------------------------------------------

\section{Introdução}
\begin{frame}{Panorama Atual do Mercado de Jogos}

    \begin{columns}{}
        \begin{column}{0.6\textwidth}
            \begin{figure}
                \centering
                \caption{Global games market: growth drivers and challenges for 2025-2027}
                \includegraphics[width=6cm]{images/games_2015_2027_revenue.png}
                \source{Newzoo}
            \end{figure}
        \end{column}
        \begin{column}{0.4\textwidth}
            \justify
            \small{O gráfico demonstra o crescimento constante do mercado global de
                jogos de 2015 a 2027, com uma previsão de aumento de 3.7\% da sua
                receita entre 2025 e 2027. Com uma renda prevista de \textbf{\$198
                    bilhões} em 2027, evidencia a importância desse mercado na economia
                atual.}
        \end{column}
    \end{columns}
\end{frame}

\begin{frame}{Crescimentos dos Jogos Roguelikes}
    \begin{columns}{}
        \begin{column}{0.6\textwidth}
            \begin{figure}
                \centering
                \caption{Balatro (2024)}
                \includegraphics[width=6cm]{images/balatro.jpg}
                \source{Wikipedia}
            \end{figure}
        \end{column}
        \begin{column}{0.4\textwidth}
            \justify
            \small{Balatro é um roguelike de construção de baralhos. \textbf{Foi um dos
                    6 jogos indicados para jogo do ano no The Game Awards 2024}. Isso
                mostra que, mesmo sendo um gênero antigo, ainda possui grande
                destaque no mercado de jogos.}
        \end{column}
    \end{columns}
\end{frame}

\begin{frame}{Crescimentos dos Jogos Roguelikes}
    \begin{figure}
        \centering
        \caption{Hell Clock (2025): Roguelike brasileiro inspirado na Guerra dos Canudos}
        \includegraphics[width=10cm]{images/hellclock.png}
        \source{Steam}
        \label{fig:ufc_emblem}
    \end{figure}
\end{frame}


\begin{frame}{Geração Procedural de Conteúdo (PCG)}
    \begin{figure}
        \centering
        \caption{Mapas gerados utilizando PCG (The Binding of Isaac)}
        \includegraphics[width=9cm]{images/bind_of_issac_maps.png}
        \source{Level Generation by Joining Geometry \cite{e2020procedural}}
    \end{figure}
    \small{Uma das principais características dos jogos Roguelike é a presença
        de conteúdos gerados proceduralmente. Ou seja, conteúdos gerados por
        algoritmos, geralmente utilizam geradores de números pseudo aleatórios.}
\end{frame}

\begin{frame}{Geração Procedural de Conteúdo (PCG)}
    \begin{columns}{}
        \begin{column}{0.6\textwidth}
            \begin{figure}
                \centering
                \caption{Gerador de Mapas com Diversos Parâmetros de Configuração}
                \includegraphics[width=6cm]{images/pcg_config.png}
                \source{Regressão para Predição de Mapas Gerados Proceduralmente. \cite{marchandaplicaccao}}
            \end{figure}
        \end{column}
        \begin{column}{0.4\textwidth}
            \justify
            \small{A geração desses conteúdos geralmente pode ser controlada por
                parâmetros como tamanho do objeto gerado, quantidades de elementos ou até
                sementes para o gerador pseudo aleatório.}
        \end{column}
    \end{columns}
\end{frame}

\begin{frame}{PCG + LLMs}
    \begin{figure}
        \centering
        \caption{Geração de Mapas Utilizando Descrições Textuais}
        \includegraphics[width=8cm]{images/prompt_to_level.png}
        \source{MarioGPT: Open-Ended Text2Level Generation through Large Language Models \cite{sudhakaran2023mariogpt}}
    \end{figure}
    \small{Com o surgimento dos Modelos Transformers Generativos, surgem grandes
        oportunidades de integrar essa tecnologia na geração procedural de conteúdo
        em jogos. Permiteindo que o controle da geração seja feito por prompts
        textuais, tornando a geração mais expressiva do que a configuração de
        parâmetros predefinidos. }
\end{frame}

\begin{frame}{Objetivo Geral}
    \justify
    Desenvolver um sistema que recebe descrições textuais e, utilizando um LLM, traduz em um nível estruturado e jogável de um jogo no estilo roguelike
\end{frame}

\begin{frame}{Objetivos Específicos}
    \begin{itemize}
        \item<1-> Desenvolver um pipeline de geração que receba uma descrição
              textual como entrada e utilize um LLM para produzir uma representação de
              mapa estruturada.
        \item<2-> Investigar e aplicar técnicas de engenharia de prompt para guiar o
              LLM de forma eficaz.
        \item<3-> Implementar um protótipo funcional que integre o pipeline de
              geração, capaz de processar os dados do mapa e renderizá-los
              visualmente.
    \end{itemize}
\end{frame}

\section{Fundamentação}
\begin{frame}{Oriegem dos Roguelikes}
    \begin{columns}{}
        \begin{column}{0.6\textwidth}
            \begin{figure}
                \centering
                \caption{Rogue (1980)}
                \includegraphics[width=6cm]{images/rogue1980.png}
                \source{Wikipedia}
            \end{figure}
        \end{column}
        \begin{column}{0.4\textwidth}
            \justify
            \small {Em 1980 surge \textbf{Rogue}, um jogo de terminal onde o
                jogador explorar masmorras geradas aleatoriamente, enfrentando
                monstros, coletando itens e tentando chegar ao final em uma única
                vida.}
        \end{column}
    \end{columns}
\end{frame}


\begin{frame}{Oriegem dos Roguelikes}
    \begin{columns}{}
        \begin{column}{0.6\textwidth}
            \begin{figure}
                \centering
                \caption{Rogue (1980)}
                \includegraphics[width=6cm]{images/rogue1980.png}
                \source{Wikipedia}
            \end{figure}
        \end{column}
        \begin{column}{0.4\textwidth}
            \justify
            \small{ Esse jogo foi tão marcante que deu origem a um gênero
                inteiro chamado Roguelike, que se baseia em elementos como:}
            \begin{itemize}
                \item <2-> Geração procedural.
                \item <3-> Permadeath (morte permanente).
                \item <4-> Exploração estratégica.
                \item <5-> Alta rejogabilidade.
            \end{itemize}
        \end{column}
    \end{columns}
\end{frame}

\begin{frame}{Procedural Content Generation (PCG)}
    \begin{columns}{}
        \begin{column}{0.6\textwidth}
            \begin{figure}
                \centering
                \caption{Diferentes Métodos de PCG}
                \includegraphics[width=6cm]{images/pcg_methods.png}
                \source{antoniosliapis.com}
            \end{figure}
        \end{column}
        \begin{column}{0.4\textwidth}
            \small{Procedural Content Generator (PCG) é uma técnica utilizada
                no desenvolvimento de jogos para \textbf{criar conteúdo de forma
                    automática, por meio de algoritmos e regras predefinidas}.}
        \end{column}
    \end{columns}
\end{frame}

% \begin{frame}{PCG Exemplo Random walk}
%     Um exemplo muito simples de algoritmo de PCG é o \textbf{Random Walk}. Esse
%     algoritmo gera caminhos aleatórios que podem ser utilizados para estruturar
%     um mapa de um jogo. Ele é consiste na seguinte sequência de passos:
%     \begin{enumerate}
%         \item <2-> Começa em uma posição inicial.
%         \item <3-> Anda aleatoriamente em uma das quatro direções.
%         \item <4-> Marca os espaços por onde passa como "caminho"
%         \item <5-> Repete esse processo por um número definido de passos ou até
%               atingir uma porcentagem do mapa marcado.
%     \end{enumerate}
% \end{frame}

% \begin{frame}{PCG Exemplo Random walk}
%     \centering
%     \only<1>{
%         \begin{figure}
%             \caption{Random Walk - Parte 1}
%             \includegraphics[width=6cm]{images/randomwalk1.png}
%             \source{Wikipedia}
%         \end{figure}
%     }
%     \only<2>{
%         \begin{figure}
%             \caption{Random Walk - Parte 2}
%             \includegraphics[width=6cm]{images/randomwalk2.png}
%             \source{Wikipedia}
%         \end{figure}
%     }
%     \only<3>{
%         \begin{figure}
%             \caption{Random Walk - Parte 3}
%             \includegraphics[width=6cm]{images/randomwalk3.png}
%             \source{Wikipedia}
%         \end{figure}
%     }
%     \only<4>{
%         \begin{figure}
%             \caption{Random Walk - Parte 4}
%             \includegraphics[width=6cm]{images/randomwalk4.png}
%             \source{Wikipedia}
%         \end{figure}
%     }
%     \only<5>{
%         \begin{figure}
%             \caption{Random Walk - Parte 5}
%             \includegraphics[width=6cm]{images/randomwalk5.png}
%             \source{Wikipedia}
%         \end{figure}
%     }
%     \only<6>{
%         \begin{figure}
%             \caption{Random Walk - Parte 6}
%             \includegraphics[width=6cm]{images/randomwalk6.png}
%             \source{Wikipedia}
%         \end{figure}
%     }
% \end{frame}

\begin{frame}{Geração de texto com LLMs}
    \begin{columns}{}
        \begin{column}{0.6\textwidth}
            \begin{figure}
                \centering
                \caption{LLMs}
                \includegraphics[width=6cm]{images/llms_examples.jpg}
                \source{blog.n8n.io}
            \end{figure}
        \end{column}
        \begin{column}{0.4\textwidth}
            \justify
            Em contraste com algoritmos simples que marcam espaços bidimensionais de
            forma aleatória, temos a \textbf{geração de texto utilizando LLMs}.
        \end{column}
    \end{columns}
\end{frame}

\begin{frame}{Geração de texto com LLMs}
    \begin{columns}{}
        \begin{column}{0.6\textwidth}
            \begin{figure}
                \centering
                \caption{Probabilidades retornadas pela LLM}
                \includegraphics[width=6cm]{images/llm_probabilities.png}
                \source{3Blue1Brown}
            \end{figure}
        \end{column}
        \begin{column}{0.4\textwidth}
            \justify
            Em uma visão de alto nível, LLMs são \textbf{funções matemáticas
                extremamente sofisticadas} que recebem um trecho de texto como
            entrada e retornam os tokens com maior probabilidade de
            continuar esse texto de forma coerente.
        \end{column}
    \end{columns}
\end{frame}

\begin{frame}{Geração de texto com LLMs}
    \begin{columns}{}
        \begin{column}{0.6\textwidth}
            \begin{figure}
                \centering
                \caption{Geração de um Saida da LLM}
                \includegraphics[width=6cm]{images/llm_generated_text.png}
                \source{3Blue1Brown}
            \end{figure}
        \end{column}
        \begin{column}{0.4\textwidth}
            \justify
            \small{ Para gerar uma resposta, o texto é passado para LLM, que
                calcula quais palavras (ou tokens) têm maior probabilidade de vir em
                seguida. Uma dessas opções é \textbf{escolhida e anexada ao texto}.
                Esse novo texto, agora com o token adicionado, é \textbf{usado como
                    nova entrada para a LLM}. Esse processo se \textbf{repete várias
                    vezes} até que uma resposta completa seja construída.}
        \end{column}
    \end{columns}
\end{frame}

\begin{frame}{Geração de texto com LLMs}
    \begin{columns}{}
        \begin{column}{0.6\textwidth}
            \begin{figure}
                \centering
                \caption{Processo de Aprendizados das LLMs}
                \includegraphics[width=6cm]{images/llm_learn.png}
                \source{3Blue1Brown}
            \end{figure}
        \end{column}
        \begin{column}{0.4\textwidth}
            \justify
            Os modelos aprendem a fazer predições através do processamento de
            uma quantidade \textbf{extremamente grande de texto, geralmente adquirido da
                internet}.
        \end{column}
    \end{columns}
\end{frame}

\section{Trabalhos Relacionados}
\begin{frame}{Word2World}
    \begin{figure}
        \centering
        \caption{Pipeline do Projeto Word2World}
        \includegraphics[width=9cm]{images/word2world.png}
        \source{Word2World: Generating Stories and Worlds through Large Language Models \cite{nasir2024word2world}}
    \end{figure}
    \only<1>{
        \small{Tanto o trabalho Word2World quanto esse TCC definem um pipeline
            para a geração de mapas utilizando LLMs sem a necessidade de fine tune
            prévio.}
    }
    \only<2>{
        \small{ A principal diferença está que no Word2World, a LLM cria a
            própria história, ou seja, o usuário não tem controle da geração. Além
            disso, o TCC visa a geração de mapas para jogos Roguelike, enquanto o
            Word2World não gera mapas para um gênero específico.}
    }
\end{frame}

\begin{frame}{MarioGPT}
    \begin{figure}
        \centering
        \caption{Pipeline do Projeto MarioGPT}
        \includegraphics[width=9cm]{images/mario_gpt.png}
        \source{MarioGPT: Open-Ended Text2Level Generation through Large Language Models\cite{sudhakaran2023mariogpt}}
    \end{figure}
    \only<1>{
        \small{MarioGPT tem o mesmo objetivo de gerar mapas baseados em
            descrições textuais, mas utiliza uma versão ajustada do GPT 2. A
            utilização de fine-tuning traz a necessidade de uma grande amostra de
            treino. No caso do MarioGPT foram utilizadas 200.000 amostras de mapas.}
    }
    \only<2>{
        \small{A proposta do TCC remove a necessidade de fine-tuning ao se
            utilizar de técnicas de engenharia de prompt. Sem contar a diferença de
            gêneros, o MarioGPT gera mapas de platformers, enquanto o TCC se propõe
        }
    }
\end{frame}

\begin{frame}{MarioGPT}
    \begin{figure}
        \centering
        \caption{Pipeline do Projeto MarioGPT}
        \includegraphics[width=9cm]{images/mario_gpt.png}
        \source{MarioGPT: Open-Ended Text2Level Generation through Large Language Models\cite{sudhakaran2023mariogpt}}
    \end{figure}
    \only<1>{
        \small{MarioGPT tem o mesmo objetivo de gerar mapas baseados em
            descrições textuais, mas utiliza uma versão ajustada do GPT 2. A
            utilização de fine-tuning traz a necessidade de uma grande amostra de
            treino. No caso do MarioGPT foram utilizadas 200.000 amostras de mapas.}
    }
    \only<2>{
        \small{A proposta do TCC remove a necessidade de fine-tuning ao se
            utilizar de técnicas de engenharia de prompt. Sem contar a diferença de
            gêneros, o MarioGPT gera mapas de platformers, enquanto o TCC se propõe
            a gerar mapas de Roguelikes.  }
    }
\end{frame}

\begin{frame}{Game Environment Design Creator}
    \begin{figure}
        \centering
        \caption{Pipeline e Interface do Projeto}
        \includegraphics[width=7cm]{images/design_creator.png}
        \source{Game Environment Design Creator Using Artificial Intelligence Procedural Generation \cite{auxtero2023game}}
    \end{figure}
    \only<1>{
        \small{O trabalho relacionado busca desenvolver uma ferramenta para
            auxiliar pessoas sem conhecimento de game design e programação a criarem
            jogos. O usuário escreve uma descrição e a LLM cria uma versão inicial
            da sala, o que se assemelha com o objetivo text-to-level do TCC.}
    }
    \only<2>{
        \small{Esse trabalho relacionado foca na experiência do usuário.
            Enquanto esse TCC foca no pipeline de geração. Além disso, o trabalho
            relacionado cria pequenas salas para jogos de visão de cima e não possui
            um foco em um gênero específico.  }
    }
\end{frame}

\begin{frame}{LLMaker}
    \begin{columns}{}
        \begin{column}{0.6\textwidth}
            \begin{figure}
                \centering
                \caption{Interface do LLMaker}
                \includegraphics[width=6cm]{images/llmaker.png}
                \source{Consistent Game Content Creation via Function Calling for Large Language Models \cite{gallotta2024consistent}}
            \end{figure}
        \end{column}
        \begin{column}{0.4\textwidth}
            \justify
            \only<1>{
                \small{O objetivo desse projeto é desenvolver uma ferramenta de design onde o
                    usuário conversa em um chat LLM e então a LLM converte a linguagem
                    natural em alterações no mapa.}
            }
            \only<2>{
                \small{Uma semelhança ao TCC está na geração de mapas com LMMs para
                    Dungeon Crawlers, gênero com estrutura muito parecida com os Roguelikes.
                    Entretanto, o foco desse trabalho relacionado é converter texto do usuário e
                    modificações no mapa gerado.}
            }
        \end{column}
    \end{columns}
\end{frame}

\section{Metodologia}
\begin{frame}{Metodologia do Trablaho}
    \begin{figure}
        \centering
        \caption{Fluxograma da Metodologia}
        \includegraphics[width=10cm]{images/metodologia.png}
        \source{Elaborado pelo autor}
    \end{figure}
\end{frame}

\begin{frame}{Definição da Arquitetura do Gerador de Mapas}
    \begin{figure}
        \centering
        \caption{Arquitetura Híbrida proposta}
        \includegraphics[width=7cm]{images/geral_map_generetor_flux.png}
        \source{Elaborado pelo autor}
    \end{figure}
    \small{ Conforme demonstrado por \citeauthor{yan2023inherent}, modelos de
        estado da arte, como o \textit{GPT-4}, apresentam dificuldades substanciais
        em tarefas que exigem compreensão precisa de coordenadas, como plotar pontos
        em espaços 2D/3D ou executar algoritmos de busca de caminho.  }
\end{frame}

\begin{frame}{Implementação do Gerador de Assets}
    \begin{figure}
        \centering
        \caption{Structured Outputs}
        \includegraphics[width=7cm]{images/structured_outputs.png}
        \source{humanloop.com}
    \end{figure}
    \small{Um dos desafios recorrentes na utilização de LLMs em sistemas
        de software é a natureza não determinística e não estruturada das saídas
        textuais. Para mitigar esse problema e garantir que o Gerador de Assets
        produza objetos de jogo válidos, empregou-se o conceito de
        \textit{Structured Outputs}.}
\end{frame}

\begin{frame}{Implementação do Gerador de Assets}
    \begin{figure}
        \centering
        \caption{RAG}
        \includegraphics[width=7cm]{images/rag.png}
        \source{medium.com/@drjulija}
    \end{figure}
    \small{Utiliza-se o RAG para integrar a capacidade gerativa de LLMs à
        recuperação de ativos visuais em bancos vetoriais, viabilizando a geração
        procedural de mundos 2D baseada em semântica.}
\end{frame}

\begin{frame}{Implementação do Gerador de Assets}
    \begin{figure}
        \centering
        \caption{Kenney 1-Bit Pack}
        \includegraphics[width=7cm]{images/kenney-1-bit-pack.png}
        \source{kenney.nl}
    \end{figure}
    \small{A base visual do sistema foi construída a partir do \textit{"1-Bit
            Pack"} disponibilizado pelo site \textit{Kenney} sob licença
        \textit{Creative Commons CC0}. A escolha deste conjunto deve-se à sua alta
        diversidade temática}
\end{frame}

\begin{frame}{Implementação do Gerador de Assets}
    \only<1>{
        \begin{itemize}
            \item \textbf{Entities} (88 descrições): Inclui inimigos, NPCs e o
                  avatar do jogador.
            \item \textbf{Items} (183 descrições): Abrange poções, armas, armaduras,
                  chaves e tesouros.
            \item \textbf{Environments} (218 descrições): Contém paredes, pisos,
                  obstáculos naturais e vegetação.
        \end{itemize}
    }
    \only<2>{
        \begin{figure}
            \centering
            \caption{Funções de Query do Banco}
            \includegraphics[width=10cm]{images/query_vector_store.png}
            \source{Elaborado pelo autor}
        \end{figure}
    }
\end{frame}

\begin{frame}{Implementação do Gerador de Assets}
    \begin{figure}
        \centering
        \caption{Fluxograma de Geração de um Pacote de Assets}
        \includegraphics[width=11cm]{images/pipeline_geracao_bundle.png}
        \source{Elaborado pelo autor}
    \end{figure}
\end{frame}

\begin{frame}{Implementação do Gerador de Assets}
    \begin{figure}
        \centering
        \caption{Tecnologias Utilizadas no Gerador de Assets}
        \includegraphics[width=11cm]{images/assets_generator_technologies.png}
        \source{Elaborado pelo autor}
    \end{figure}
\end{frame}

\begin{frame}[shrink=12]{Implementação do Gerador de Mapas}
    \begin{algorithm}[H]
        \caption{Geração Procedural da Estrutura da Masmorra}
        \label{alg:dungeon_gen_high_level}
        \SetAlgoLined

        % Definições locais para garantir funcionamento (se já estiver no preâmbulo, pode remover)
        \SetKwInput{Entrada}{Entrada}
        \SetKwInput{Saida}{Saída}
        \SetKwBlock{Inicio}{Início}{fim}
        \SetKwFor{Para}{para}{faça}{fim}
        \SetKwIF{Se}{SenaoSe}{Senao}{se}{então}{senão se}{senão}{fim}
        \SetKw{Retorna}{retorna}
        \SetKw{Continuar}{continuar}
        % -----------------------------------------------------------

        \Entrada{Restrições Geométricas ($D_{mapa}$), Limite de Tentativas ($N_{max}$)}
        \Saida{Topologia da Masmorra ($Salas$)}

        \Inicio{
            $Salas \leftarrow \emptyset$\;

            \Para{$i \leftarrow 0$ \textbf{até} $N_{max}$}{
                $Candidata \leftarrow$ GerarRetanguloAleatorio($D_{mapa}$)\;

                \Se{HaIntersecao($Candidata$, $Salas$)}{
                    \Continuar\;
                }

                EscavarNoMapa($Candidata$)\;

                \Se{$Salas$ não está vazia}{
                    $Alvo \leftarrow$ ObterUltimaSala($Salas$)\;
                    CriarCorredorConectando($Alvo$, $Candidata$)\;
                }

                Adicionar $Candidata$ ao conjunto $Salas$\;
            }

            \Retorna $Salas$\;
        }
    \end{algorithm}
\end{frame}

\begin{frame}{Implementação do Gerador de Mapas}
    \begin{figure}
        \centering
        \caption{Tecnologias Utilizadas no Gerador de Jogos}
        \includegraphics[width=6cm]{images/godot.jpg}
        \source{Elaborado pelo autor}
    \end{figure}
\end{frame}

\begin{frame}{Definição das Métricas de Validação}
    \begin{itemize}
        \item<1-> \textbf{Coerência}: Uma LLM atua como juíza, atribuindo uma nota
              de 0 a 100 ao avaliar o alinhamento lógico entre a descrição textual
              detalhada e os objetos (JSON) gerados para o jogo.
        \item<2-> \textbf{Reconstrução Semântica}: Testa se os objetos gerados
              comunicam o tema original; uma LLM tenta recriar a descrição baseando-se
              apenas nos ativos, comparando o resultado ao texto original via
              similaridade de cosseno.
    \end{itemize}
\end{frame}

\section{Resultados}

\begin{frame}{Configuração Experimental}
    \begin{itemize}
        \item Configurações do Gerador de Assets:
              \begin{itemize}
                  \item \textbf{Níveis da Masmorra:} 6 níveis de profundidade.
                  \item \textbf{Inimigos:} 20 variações de inimigos.
                  \item \textbf{Armas:} 30 tipos de armas.
              \end{itemize}
    \end{itemize}
\end{frame}

\begin{frame}{Configuração Experimental}
    \begin{itemize}
        \item Modelos escolhidos para geração:
              \begin{itemize}
                  \item \textbf{Llama-4 Maverick (17B)}: Modelo massivo (400B totais) especializado em seguir instruções.
                  \item \textbf{GPT-OSS-120B}: Modelo de grande porte para referência de alta performance.
                  \item \textbf{GPT-OSS-20B}: Modelo reduzido e eficiente, ideal para rodar em computadores locais.
              \end{itemize}
        \item \textbf{text-embedding-004}: Modelo de \textit{embeddings} usado
              no \textbf{RAG} e no cálculo da métrica de \textbf{Reconstrução Semântica}.
        \item \textbf{Gemini 3 Pro}: Modelo usado como "Juiz" na métrica de \textbf{Coerência}.
    \end{itemize}
\end{frame}

\begin{frame}{Configuração Experimental}
    Os testes foram conduzidos sobre cinco \textit{prompts} (P1 a P5) abrangendo
    temas distintos, comuns ao gênero \textit{roguelike}:
    \begin{itemize}
        \item \textbf{P1:} \textit{The Cursed Dwarven Forge} (Medieval/Fantasia).
        \item \textbf{P2:} \textit{The Derelict Starship} (Ficção Científica/Espaço).
        \item \textbf{P3:} \textit{The Smuggler's Grotto} (Pirata/Náutico).
        \item \textbf{P4:} \textit{Neon Skyline Penthouse} (Cyberpunk/Urbano).
        \item \textbf{P5:} \textit{The Living Hive} (Alien/Bio-Horror).
    \end{itemize}
\end{frame}

\begin{frame}{Análise Quantitativa}
    \begin{figure}
        \centering
        \caption{Resultados da Métrica de Coerência (0-100)}
        \includegraphics[width=10cm]{images/coerencia.png}
        \source{Elaborado pelo autor}
    \end{figure}
\end{frame}

\begin{frame}{Análise Quantitativa}
    \begin{figure}
        \centering
        \caption{Resultados da Métrica de Reconstrução (Similaridade de Cosseno)}
        \includegraphics[width=10cm]{images/reconstrucao.png}
        \source{Elaborado pelo autor}
    \end{figure}
\end{frame}

\begin{frame}{Análise Qualitativa (GPT-120b com P2, tema Sci-Fi)}
    A Void-Strider Ark é uma colossal nave mineradora à deriva, envolta em uma
    atmosfera industrial sombria e decadente, onde a inteligência artificial
    AURORA, corrompida após uma anomalia gravitacional, transformou sistemas de
    segurança e drones em carrascos implacáveis contra qualquer presença
    orgânica. Nesse labirinto de metal e sombras, o jogador deve enfrentar
    riscos letais como vazamentos de vácuo, radiação e armadilhas elétricas,
    navegando por ambientes de estética e iluminação severa para subverter a
    lógica distorcida da IA, alcançar a ponte de comando e retomar o controle do
    reator central.
\end{frame}

\begin{frame}{Análise Qualitativa (GPT-120b com P2, tema Sci-Fi)}
    \begin{figure}
        \centering
        \caption{Player, Níveis e Objetivos}
        \includegraphics[height=6cm]{images/prompt2_gpt_120b_levels_player_and_final_objective.png}
        \source{Elaborado pelo autor}
    \end{figure}
\end{frame}

\begin{frame}{Análise Qualitativa (GPT-120b com P2, tema Sci-Fi)}
    \begin{figure}
        \centering
        \caption{Inimigos}
        \includegraphics[height=6cm]{images/prompt2_gpt_120b_enemies.png}
        \source{Elaborado pelo autor}
    \end{figure}
\end{frame}

\begin{frame}{Análise Qualitativa (GPT-120b com P2, tema Sci-Fi)}
    \begin{figure}
        \centering
        \caption{Armas}
        \includegraphics[height=6cm]{images/prompt2_gpt_120b_weapons.png}
        \source{Elaborado pelo autor}
    \end{figure}
\end{frame}

\section{Conclusão}
\begin{frame}{Conclusão}
\end{frame}


%% ---------------------------------------------------------------------------
% Reference frames
\begin{frame}[allowframebreaks]
    \frametitle{Referências}
    \printbibliography
\end{frame}

%% ---------------------------------------------------------------------------
% Final frame
\begin{frame}{}
    \centering
    \huge{\textbf{\example{Obrigado(a) pela Atenção!}}}
\end{frame}

\end{document}
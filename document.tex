%%%%%%%%%%%%%%%%%%%%%%%%%%%%%%%%%%%%%%%%%%%%%%%%%%%%%%%%%%%%%%%%%%%%%%%%%%%%%%%%%%%
%% This project aims to create the UFC template for presentation.                %%
%% author: Maurício Moreira Neto - Doctoral student in Computer Science (MDCC)   %%
%% contacts:                                                                     %%
%%    e-mail: maumneto@ufc.br                                                    %%
%%    linktree: https://linktr.ee/maumneto                                       %%
%%%%%%%%%%%%%%%%%%%%%%%%%%%%%%%%%%%%%%%%%%%%%%%%%%%%%%%%%%%%%%%%%%%%%%%%%%%%%%%%%%%
\documentclass{libs/ufc_format}
% Inserting the preamble file with the packages
\input{libs/preamble.tex}
% Inserting the references file
\bibliography{references.bib}

% Title
% \title[Geração de Mapas com LLMs]{\huge\textbf{Geração de Mapas com LLMs para Roguelikes}}
% Subtitle
\subtitle{Geração de Mapas para Jogos Roguelike a Partir de Descrições Textuais
Utilizando Modelos de Linguagem de Larga Escala }
% Author of the presentation
\author{Gustavo Gurgel (Orientador: Cristiano Bacelar)}
% Institute's Name
\institute[UFC]{
    % email for contact
    \normalsize{\email{gusgurgel@alu.ufc.br}}
    \newline
    % Department Name
    % \department{}
    % \newline
    % university name
    \ufc
}
% date of the presentation
\date{\today}


%%%%%%%%%%%%%%%%%%%%%%%%%%%%%%%%%%%%%%%%%%%%%%%%%%%%%%%%%%%%%%%%%%%%%%%%%%%%%%%%%%
%% Start Document of the Presentation                                           %%               
%%%%%%%%%%%%%%%%%%%%%%%%%%%%%%%%%%%%%%%%%%%%%%%%%%%%%%%%%%%%%%%%%%%%%%%%%%%%%%%%%%
\begin{document}
% insert the code style
\input{libs/code_style}

\begin{frame}{}
    \maketitle
\end{frame}

\begin{frame}{Sumário}
    \begin{multicols}{2}
        \tableofcontents
    \end{multicols}
\end{frame}

%% ---------------------------------------------------------------------------

\section{Introdução}
\begin{frame}{Escopo Geral do Problema}
    \begin{itemize}
        \item<1-> Segundo levantamentos da Newzoo, o mercado de jogos irá arrecadar
              \textbf{\$198 bilhões} no ano de 2027.
        \item<2-> Nesse mercado, os \textit{Roguelikes}, um gênero muito antigo
              (1980), continua recebendo alto reconhecimento. Dentro jogos desse
              gênero temos:
              \begin{itemize}
                  \item<3-> \textbf{Balatro (2024)}: Concorreu como um dos 6 melhores
                        jogos de 2024.
                  \item<4-> \textbf{Hell Clock (2025)}: Roguelike brasileiro inspirado na
                        Guerra dos Canudos
              \end{itemize}
        \item<5-> Um característica predominate nos \textit{Roguelikes} é a
              geração de conteúdos (items, missões, inimigos) através de algoritmos.
        \item<6-> Em geral a geração desses conteúdos é guiada através de
              \textbf{parâmetros numéricos, muitas vezes de difícil interpretação}.
    \end{itemize}
\end{frame}

\begin{frame}{Solução PCG + LLMs}
    \begin{itemize}
        \item <1-> Os grandes modelos de linguagem oferecem alta capacidade de
              interpretação de linguagem natural.
        \item <2-> Assim, destaca-se a oportunidade utilizar LLMs em sistemas de
              geração de conteúdo para jogos.
        \item <3-> Dessa forma, permitindo um controle mais livre. Em vez de
              inserir valores numéricos, o utilizador descreve o conteúdo a ser gerado
              utilizando linguagem natural.
    \end{itemize}
\end{frame}

\begin{frame}{Objetivos}
    \justify
    Desenvolver um sistema que recebe descrições textuais e, utilizando um LLM,
    traduz em um nível estruturado e jogável de um jogo no gênero
    \textit{Roguelike}.
    \begin{itemize}
        \item<2-> Desenvolver um pipeline de geração que receba uma descrição
              textual e produzir uma representação de mapa estruturada.
        \item<3-> Investigar e aplicar técnicas de engenharia de prompt para
              guiar o LLM de forma eficaz.
        \item<4-> Implementar um protótipo capaz de processar os dados do mapa e
              renderizá-los visualmente.
    \end{itemize}
\end{frame}

\section{Fundamentação}
\begin{frame}{Origem dos \textit{Roguelikes}}
    \begin{columns}{}
        \begin{column}{0.5\textwidth}
            \begin{figure}
                \centering
                \caption{Rogue (1980)}
                \includegraphics[width=6cm]{images/rogue1980.png}
                \source{Wikipedia}
            \end{figure}
        \end{column}
        \begin{column}{0.5\textwidth}
            \begin{itemize}
                \item Em 1980 surge \textbf{Rogue}, um jogo de terminal onde o
                      jogador explorar masmorras geradas aleatoriamente, enfrentando
                      monstros, coletando itens e tentando chegar ao final em uma única
                      vida.
                \item Esse jogo foi tão marcante que deu origem a um gênero
                      inteiro chamado Roguelike, que se baseia em elementos como:
                      \begin{itemize}
                          \item <2-> Geração procedural.
                          \item <3-> Permadeath (morte permanente).
                          \item <4-> Exploração estratégica.
                          \item <5-> Alta rejogabilidade.
                      \end{itemize}
            \end{itemize}
        \end{column}
    \end{columns}
\end{frame}

\begin{frame}{Geração de texto com LLMs}
    \begin{itemize}
        \item<1-> Outro exemplo de geração de conteúdo é a geração de texto
              utilizando LLMs.
        \item<2-> Fatores como definição da arquitetura
              \textit{Transformer}, maior capacidade computacional e volumes
              gigantescos de dados textuais permitiram o surgimento de Modelos
              de Linguagem com capacidades notáveis no processamento de
              linguagem natural.
    \end{itemize}
\end{frame}

\section{Trabalhos Relacionados}
\begin{frame}{Trabalhos Relacionaos}
    Utilizando a proficiência dos LLMs no processamento de linguagem natural, os
    seguintes trabalhos exploram a aplicação desses modelos na geração de mapas:
    \begin{figure}
        \centering
        \caption{Trabalhos Relacionados}
        \includegraphics[width=11cm]{images/trabalhos_relacionados.png}
        \source{Elaborado pelo autor}
    \end{figure}
\end{frame}

\section{Metodologia}
\begin{frame}{Metodologia do Trabalho}
    \begin{figure}
        \centering
        \caption{Fluxograma da Metodologia}
        \includegraphics[width=11cm]{images/metodologia.png}
        \source{Elaborado pelo autor}
    \end{figure}
\end{frame}

\begin{frame}{Definição da Arquitetura do Gerador de Mapas}
    \begin{figure}
        \centering
        \caption{Arquitetura Híbrida proposta}
        \includegraphics[width=7cm]{images/geral_map_generetor_flux.png}
        \source{Elaborado pelo autor}
    \end{figure}
    \small{ Conforme demonstrado por \citeauthor{yan2023inherent}, modelos de
        estado da arte, como o \textit{GPT-4}, apresentam dificuldades substanciais
        em tarefas que exigem compreensão precisa de coordenadas, como plotar pontos
        em espaços 2D/3D ou executar algoritmos de busca de caminho.  }
\end{frame}

\begin{frame}{Implementação do Gerador de Assets}
    \begin{itemize}
        \item<1-> \textbf{Problema 1}: Saídas não estruturadas dos LLMs não são
              processáveis pelo Gerador de Mapas.
        \item<1-> \textbf{Solução}: Utilização da técnica de \textit{Structured
                  Outputs}.  Uso modelos ajustados para responder seguindo um esquema
              JSON.
    \end{itemize}
    \vspace{0.5cm}
    \begin{itemize}
        \item<2-> \textbf{Problema 2}: O LLM gera saídas textuais, porém o jogo precisa
              de ativos visuais para representar items, inimigos e ambientes.
        \item<2-> \textbf{Solução}: Utilização da técnica de \textit{RAG}. Um banco
              vetorial guarda tuplas de descrições e images. O código pode utilizar
              saídas textuais do LLM para recuperar imagens.
    \end{itemize}
\end{frame}

\begin{frame}{Implementação do Gerador de Assets}
    \begin{figure}
        \centering
        \caption{Funções de Query do Banco}
        \includegraphics[width=8cm]{images/query_vector_store.png}
        \source{Elaborado pelo autor}
    \end{figure}
    \small{Utilizando o pacote de tiles "Kenney 1-Bit Pack"\ foram feitos 3
        bancos vetoriais: Entities (88 items), Items (183 items) e
        Environment(218 items)}
\end{frame}

\begin{frame}{Implementação do Gerador de Assets}
    \begin{figure}
        \centering
        \caption{Fluxograma de Geração de um Pacote de Assets}
        \includegraphics[width=11cm]{images/pipeline_geracao_bundle.png}
        \source{Elaborado pelo autor}
    \end{figure}
\end{frame}

\begin{frame}{Implementação do Gerador de Assets}
    \begin{figure}
        \centering
        \caption{Tecnologias Utilizadas no Gerador de Assets}
        \includegraphics[width=11cm]{images/assets_generator_technologies.png}
        \source{Elaborado pelo autor}
    \end{figure}
\end{frame}

\begin{frame}[shrink=12]{Implementação do Gerador de Mapas}
    \begin{algorithm}[H]
        \caption{Geração Procedural da Estrutura da Masmorra}
        \label{alg:dungeon_gen_high_level}
        \SetAlgoLined

        % Definições locais para garantir funcionamento (se já estiver no preâmbulo, pode remover)
        \SetKwInput{Entrada}{Entrada}
        \SetKwInput{Saida}{Saída}
        \SetKwBlock{Inicio}{Início}{fim}
        \SetKwFor{Para}{para}{faça}{fim}
        \SetKwIF{Se}{SenaoSe}{Senao}{se}{então}{senão se}{senão}{fim}
        \SetKw{Retorna}{retorna}
        \SetKw{Continuar}{continuar}
        % -----------------------------------------------------------

        \Entrada{Restrições Geométricas ($D_{mapa}$), Limite de Tentativas ($N_{max}$)}
        \Saida{Topologia da Masmorra ($Salas$)}

        \Inicio{
            $Salas \leftarrow \emptyset$\;

            \Para{$i \leftarrow 0$ \textbf{até} $N_{max}$}{
                $Candidata \leftarrow$ GerarRetanguloAleatorio($D_{mapa}$)\;

                \Se{HaIntersecao($Candidata$, $Salas$)}{
                    \Continuar\;
                }

                EscavarNoMapa($Candidata$)\;

                \Se{$Salas$ não está vazia}{
                    $Alvo \leftarrow$ ObterUltimaSala($Salas$)\;
                    CriarCorredorConectando($Alvo$, $Candidata$)\;
                }

                Adicionar $Candidata$ ao conjunto $Salas$\;
            }

            \Retorna $Salas$\;
        }
    \end{algorithm}
\end{frame}

\begin{frame}{Implementação do Gerador de Mapas}
    \begin{figure}
        \centering
        \caption{Tecnologias Utilizadas no Gerador de Jogos}
        \includegraphics[width=6cm]{images/godot.jpg}
        \source{Elaborado pelo autor}
    \end{figure}
\end{frame}

\begin{frame}{Definição das Métricas de Validação}
    \begin{itemize}
        \item<1-> \textbf{Coerência}: Uma LLM atua como juíza, atribuindo uma nota
              de 0 a 100 ao avaliar o alinhamento lógico entre a descrição textual
              detalhada e os objetos (JSON) gerados para o jogo.
        \item<2-> \textbf{Reconstrução Semântica}: Testa se os objetos gerados
              comunicam o tema original; uma LLM tenta recriar a descrição baseando-se
              apenas nos ativos, comparando o resultado ao texto original via
              similaridade de cosseno.
    \end{itemize}
\end{frame}

\section{Resultados}

\begin{frame}{Configuração Experimental}
    \begin{itemize}
        \item<1-> Modelos escolhidos para geração:
              \begin{itemize}
                  \item<1-> \textbf{Llama-4 Maverick (17B)}: Modelo massivo (400B totais) especializado em seguir instruções.
                  \item<1-> \textbf{GPT-OSS-120B}: Modelo de grande porte para referência de alta performance.
                  \item<1-> \textbf{GPT-OSS-20B}: Modelo reduzido e eficiente, ideal para rodar em computadores locais.
              \end{itemize}
        \item<2-> \textbf{text-embedding-004}: Modelo de \textit{embeddings} usado
              no \textbf{RAG} e no cálculo da métrica de \textbf{Reconstrução Semântica}.
        \item<3-> \textbf{Gemini 3 Pro}: Modelo usado como "Juiz" na métrica de \textbf{Coerência}.
    \end{itemize}
\end{frame}

\begin{frame}{Configuração Experimental}
    \only<1>{
    Configurações do Gerador de Assets:
    \begin{itemize}
        \item \textbf{Níveis da Masmorra:} 6 níveis de profundidade.
        \item \textbf{Inimigos:} 20 variações de inimigos.
        \item \textbf{Armas:} 30 tipos de armas.
    \end{itemize}
    }
    \vspace{0.5cm}
    Os testes foram conduzidos sobre cinco \textit{prompts} (P1 a P5) abrangendo
    temas distintos, comuns ao gênero \textit{roguelike}:
    \begin{itemize}
        \item \textbf{P1:} \textit{The Cursed Dwarven Forge} (Medieval/Fantasia).
        \item \textbf{P2:} \textit{The Derelict Starship} (Ficção Científica/Espaço).
        \item \textbf{P3:} \textit{The Smuggler's Grotto} (Pirata/Náutico).
        \item \textbf{P4:} \textit{Neon Skyline Penthouse} (Cyberpunk/Urbano).
        \item \textbf{P5:} \textit{The Living Hive} (Alien/Bio-Horror).
    \end{itemize}
\end{frame}

\begin{frame}{Análise Quantitativa}
    \begin{figure}
        \centering
        \caption{Resultados da Métrica de Coerência (0-100)}
        \includegraphics[width=8cm]{images/coerencia.png}
        \source{Elaborado pelo autor}
    \end{figure}

    \begin{figure}
        \centering
        \caption{Resultados da Métrica de Reconstrução (Similaridade de Cosseno)}
        \includegraphics[width=8cm]{images/reconstrucao.png}
        \source{Elaborado pelo autor}
    \end{figure}
\end{frame}

\begin{frame}{Análise Qualitativa (GPT-120b com P2, tema Sci-Fi)}
    \begin{itemize}
        \item<1-> A \textbf{Void-Strider Ark} é uma colossal nave mineradora à deriva,
        envolta em uma atmosfera industrial sombria e decadente.
        \item<2-> A inteligência artificial \textbf{AURORA}, corrompida após uma
        anomalia gravitacional, transformou \textbf{sistemas de segurança e
        drones em carrascos implacáveis} contra qualquer presença orgânica.
        \item<3-> O jogador deve enfrentar riscos letais, alcançar a \textbf{ponte
        de comando} e \textbf{retomar o controle do reator central}.
    \end{itemize}
\end{frame}

\begin{frame}{Análise Qualitativa (GPT-120b com P2, tema Sci-Fi)}
    \begin{figure}
        \centering
        \caption{Player, Níveis e Objetivos}
        \includegraphics[height=6cm]{images/prompt2_gpt_120b_levels_player_and_final_objective.png}
        \source{Elaborado pelo autor}
    \end{figure}
\end{frame}

\begin{frame}{Análise Qualitativa (GPT-120b com P2, tema Sci-Fi)}
    \begin{figure}
        \centering
        \caption{Armas}
        \includegraphics[height=6cm]{images/prompt2_gpt_120b_weapons_and_items.png}
        \source{Elaborado pelo autor}
    \end{figure}
\end{frame}

\begin{frame}{Roguelike}
    \begin{columns}{}
        \begin{column}{0.6\textwidth}
            \begin{figure}
                \centering
                \caption{Visão Geral da Interface}
                \includegraphics[height=6cm]{images/full_vizion_of_the_game.png}
            \end{figure}
        \end{column}
        \begin{column}{0.5\textwidth}
            \justify
            Utilizando a \textit{game engine} Godot em sua versão 4.5.1, foi
            desenvolvido um jogo \textit{Roguelike} \textbf{capaz de processar um
                \textit{AssetBundle} em formato JSON e transformá-lo em um mapa
                jogável}.
        \end{column}
    \end{columns}
\end{frame}

\begin{frame}{Roguelike}
    \begin{figure}
        \centering
        \caption{Visão Completa de 3 Níveis Gerados}
        \includegraphics[width=11cm]{images/three_diferent_dungeons_generations.png}
        \source{Elaborado pelo autor}
    \end{figure}
\end{frame}

\section{Conclusão}
\begin{frame}{Conclusão e Contribuições}
    \begin{itemize}
        \item<1-> \textbf{Arquitetura Híbrida de Sucesso}: A solução proposta
              resolve a limitação de raciocínio espacial dos LLMs ao separar as
              responsabilidades: algoritmos tradicionais cuidam da topologia (mapas)
              enquanto o LLM atua como agente criativo (narrativa e itens).
        \item <2-> \textbf{Contribuição de Dados}: O trabalho entrega um dataset
              relevante de 489 pares de descrições e imagens (pixel art).
        \item<3-> \textbf{Eficiência Semântica e Coerência}: Através das
              métricas de Reconstrução Semântica e de Coerência, comprovou-se que o
              sistema preserva fielmente a intenção do usuário ao transformar texto em
              elementos de jogo funcionais.
    \end{itemize}
\end{frame}

\begin{frame}{Conclusão e Contribuições}
    \begin{itemize}
        \item<1-> \textbf{Viabilidade Técnica Local}: Os testes mostraram que
              modelos compactos executados localmente (20b) possuem desempenho
              comparável a modelos grandes (120b).
        \item<2-> \textbf{Democratização do Desenvolvimento}: O projeto permite
              a criação de experiências de jogo customizadas e coerentes através de
              linguagem natural.
    \end{itemize}
\end{frame}

\begin{frame}{Limitações e Trabalhos Futuros}
    \begin{itemize}
        \item<1-> \textbf{Limitação}: Os bancos vetoriais tem sua expressividade
              limitada pela quantidade de tiles disponíveis.
        \item<1-> \textbf{Possível Solução}: Modelos de difusão (como Stable
              Diffusion) para gerar texturas em tempo real
              \vspace{0.2cm}
        \item<2-> \textbf{Limitação}: O sistema apenas gera objetos unitários
              isolados.
        \item<2-> \textbf{Possível Solução}: Evoluir geração de conjuntos de
              elementos, permitindo a construção de micro-cenários mais detalhados e
              coesos.
              \vspace{0.2cm}
        \item<3-> \textbf{Limitação}: O projeto é focado apenas na geração de
              mapas para jogos Roguelikes.
              isolados.
        \item<3-> \textbf{Possível Solução}: Adaptar o sistema para outros tipos
              de jogos como RPGs, jogos de estratégia, jogos plataforma...
    \end{itemize}
\end{frame}

%% ---------------------------------------------------------------------------
% Reference frames
\begin{frame}[allowframebreaks]
    \frametitle{Referências}
    \printbibliography
\end{frame}

%% ---------------------------------------------------------------------------
% Final frame
\begin{frame}{}
    \centering
    \huge{\textbf{\example{Obrigado(a) pela Atenção!}}}
\end{frame}

\end{document}